\documentclass[a4paper, 12pt, titlepage]{article}
%=======Unpackage Things===============
\usepackage{color}
\usepackage{colortbl}
\usepackage{mathrsfs}

\usepackage{graphicx}
\usepackage{amsthm}
\usepackage{listings}
%\usepackage{fullpage}
\usepackage{epsfig}
\usepackage{amsmath}
\usepackage{latexsym}
\usepackage{amssymb}
\usepackage{amstext}
\usepackage{array}
\usepackage{titlesec}
\usepackage{float}

\titleformat{\section}
    {\normalfont\fontsize{12}{15}\bfseries}{\thesection}
    {1em}{}

%\titleformat{\subsection}
%    {\normalfont\fontsize{5}{5}\bfseries}{\thesection}
%    {1em}{}
\newcommand{\PreserveBackslash}[1]{\let\temp=\\#1\let\\=\temp}
\newcolumntype{C}[1]{>{\PreserveBackslash\centering}p{#1}}
\newcolumntype{R}[1]{>{\PreserveBackslash\raggedleft}p{#1}}
\newcolumntype{L}[1]{>{\PreserveBackslash\raggedright}p{#1}}


\begin{document}
%==title==
\title{COMP6321 Assignment 4}
\setcounter{tocdepth}{2}
\newpage
\begin{center}
    {\huge COMP6321: Assignment \#4}


    \vspace{2cm}
    Student: Qing Gu  \hspace{5cm}
    Student ID: 6935451
    \vspace{1cm}

    =================================================
\end{center}
\section{VC Dimension}
\begin{enumerate}
    \item $VC(H)=1$, since we can always shatter the subset with 1 points.

        $$-----[---\oplus-----$$
        $$-----\ominus-[--------$$
    But we can not do that when there are two points. When there is a dichotomy like this:
        $$-------\oplus---\ominus----$$

    \item $VC(H)=2$, we can find configuration for two points.
    $$----\oplus--]-----\ominus----$$
    $$----\ominus----[---\oplus----$$
    $$----\oplus-------\oplus--]--$$
    $$----\ominus-------\ominus-[---$$

    \item $VC(H)=$
        
\end{enumerate}


\end{document}
