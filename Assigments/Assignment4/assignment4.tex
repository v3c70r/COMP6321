\documentclass[a4paper, 10pt, titlepage]{article}
%=======Unpackage Things===============
\usepackage{color}
\usepackage{colortbl}
\usepackage{mathrsfs}

\usepackage{graphicx}
\usepackage{amsthm}
\usepackage{listings}
%\usepackage{fullpage}
\usepackage{epsfig}
\usepackage{amsmath}
\usepackage{latexsym}
\usepackage{amssymb}
\usepackage{amstext}
\usepackage{array}
\usepackage{titlesec}
\usepackage{float}

\titleformat{\section}
    {\normalfont\fontsize{12}{13}\bfseries}{\thesection}
    {1em}{}

%\titleformat{\subsection}
%    {\normalfont\fontsize{5}{5}\bfseries}{\thesection}
%    {1em}{}
\newcommand{\PreserveBackslash}[1]{\let\temp=\\#1\let\\=\temp}
\newcolumntype{C}[1]{>{\PreserveBackslash\centering}p{#1}}
\newcolumntype{R}[1]{>{\PreserveBackslash\raggedleft}p{#1}}
\newcolumntype{L}[1]{>{\PreserveBackslash\raggedright}p{#1}}


\begin{document}
%==title==
\title{COMP6321 Assignment 4}
\setcounter{tocdepth}{2}
\newpage
\begin{center}
    {\huge COMP6321: Assignment \#4}


    \vspace{2cm}
    Student: Qing Gu  \hspace{5cm}
    Student ID: 6935451
    \vspace{1cm}

    =================================================
\end{center}
\section{VC Dimension}
\begin{enumerate}
    \item $VC(H)=1$, since we can always shatter the subset with 1 points.
        \begin{center}
        \begin{picture}(100,60)
        \put(10,60){$-$}
        \put(0,50){\line(1,0){100}}
        \multiput(10,50)(10,0){9}{\line(0,1){5}}
        \multiput(10,40)(10,0){9}%{\makebox(0,0){\thenn\stepcounter{nn}}}
        \thicklines
        \put(20,50){\circle{3}}
        \put(80,50){\circle{3}}
        \linethickness{1pt}
        \put(50,45){\line(0,1){10}}
        \put(50,55){\line(1,0){5}}
        \put(50,45){\line(1,0){5}}
        \end{picture}
        \end{center}

        \begin{center}
        \begin{picture}(100,20)
        \put(70,60){$-$}
        \put(0,50){\line(1,0){100}}
        \multiput(10,50)(10,0){9}{\line(0,1){5}}
        \multiput(10,40)(10,0){9}%{\makebox(0,0){\thenn\stepcounter{nn}}}
        \thicklines
        \put(20,50){\circle{3}}
        \put(80,50){\circle{3}}
        \linethickness{1pt}
        \put(50,45){\line(0,1){10}}
        \put(50,55){\line(1,0){5}}
        \put(50,45){\line(1,0){5}}
        \end{picture}
        \end{center}
    But we can not do that when there are two points. When there is a dichotomy like this:
        \begin{center}
        \begin{picture}(100,60)
        \put(70,60){$-$}
        \put(10,60){$+$}
        \put(0,50){\line(1,0){100}}
        \multiput(10,50)(10,0){9}{\line(0,1){5}}
        \multiput(10,40)(10,0){9}%{\makebox(0,0){\thenn\stepcounter{nn}}}
        \thicklines
        \put(20,50){\circle{3}}
        \put(80,50){\circle{3}}
        \linethickness{1pt}
        %\put(50,45){\line(0,1){10}}
        %\put(50,55){\line(1,0){5}}
        %\put(50,45){\line(1,0){5}}
        \end{picture}
        \end{center}

    \item $VC(H)=2$, we can find configuration for two points.
        \begin{center} 
        \begin{picture}(100,60)
        \put(10,60){$-$}
        \put(70,60){$+$}
        \put(0,50){\line(1,0){100}}
        \multiput(10,50)(10,0){9}{\line(0,1){5}}
        \multiput(10,40)(10,0){9}%{\makebox(0,0){\thenn\stepcounter{nn}}}
        \thicklines
        \put(20,50){\circle{3}}
        \put(80,50){\circle{3}}
        \linethickness{1pt}
        \put(50,45){\line(0,1){10}}
        \put(50,55){\line(1,0){5}}
        \put(50,45){\line(1,0){5}}
        \end{picture}

        \begin{picture}(100,20)
        \put(10,60){$+$}
        \put(70,60){$+$}
        \put(0,50){\line(1,0){100}}
        \multiput(10,50)(10,0){9}{\line(0,1){5}}
        \multiput(10,40)(10,0){9}%{\makebox(0,0){\thenn\stepcounter{nn}}}
        \thicklines
        \put(20,50){\circle{3}}
        \put(80,50){\circle{3}}
        \linethickness{1pt}
        \put(5,45){\line(0,1){10}}
        \put(5,55){\line(1,0){5}}
        \put(5,45){\line(1,0){5}}
        \end{picture}

        \begin{picture}(100,20)
        \put(10,60){$-$}
        \put(70,60){$-$}
        \put(0,50){\line(1,0){100}}
        \multiput(10,50)(10,0){9}{\line(0,1){5}}
        \multiput(10,40)(10,0){9}%{\makebox(0,0){\thenn\stepcounter{nn}}}
        \thicklines
        \put(20,50){\circle{3}}
        \put(80,50){\circle{3}}
        \linethickness{1pt}
        \put(80,45){\line(0,1){10}}
        \put(80,55){\line(1,0){5}}
        \put(80,45){\line(1,0){5}}
        \end{picture}

        \begin{picture}(100,20)
        \put(10,60){$+$}
        \put(70,60){$-$}
        \put(0,50){\line(1,0){100}}
        \multiput(10,50)(10,0){9}{\line(0,1){5}}
        \multiput(10,40)(10,0){9}%{\makebox(0,0){\thenn\stepcounter{nn}}}
        \thicklines
        \put(20,50){\circle{3}}
        \put(80,50){\circle{3}}
        \linethickness{1pt}
        \put(30,45){\line(0,1){10}}
        \put(25,55){\line(1,0){5}}
        \put(25,45){\line(1,0){5}}
        \end{picture}
        \end{center}
        But for this configuration with 3 points, we can not shatter it.
        \begin{center}
        \begin{picture}(100,60)
        \put(10,60){$+$}
        \put(40,60){$-$}
        \put(70,60){$+$}
        \put(0,50){\line(1,0){100}}
        \multiput(10,50)(10,0){9}{\line(0,1){5}}
        \multiput(10,40)(10,0){9}%{\makebox(0,0){\thenn\stepcounter{nn}}}
        \thicklines
        \put(20,50){\circle{3}}
        \put(80,50){\circle{3}}
        \linethickness{1pt}
        \end{picture}
        \end{center}


    \item $VC(H)=$
\end{enumerate}
\section{KL-Divergence}
\begin{enumerate}
    \item There are two ways to prove it.

        \begin{itemize}
                \item Method 1:

                    Let random variable $f(x) = \frac{Q(x)}{P(x)}$ Then we can rewrite KL Divergence to the expectation of $f(x)$ whose probability is $P(x)$
                    $$KL(P||Q) = -E_p[\log{f(x)}] $$
                    $$\geq-\log{E_p[f(x)]}\mbox{~(Jensen's inequality)}$$
                    $$=-\log{\sum_xP(x)\frac{Q(X)}{P(X)}}=0$$
                    Thus, $KL(P||Q) \geq 0$.

                \item Method 2:
                    $$-KL(P||Q) = \sum_xP(x)\log{\frac{Q}{P}}$$
                    $$\because \log{x}\leq{}x-1$$
                    $$\therefore -KL(P||Q)\leq\sum_x(\frac{Q(x)}{P(x)}-1)$$
                    $$-KL(P||Q)=\sum_xQ(x)-sum_xP(x)$$
                    $$\because P(x)\mbox{~is an distribution function}$$
                    $$\therefore -KL(P||Q) \leq \sum_xQ(x)-1 \leq 0$$
                    Thus, $KL(P||Q) \geq 0$.
        \end{itemize}
    \item From above, we can observe that when $P(x) = Q(x)$, $KL(P||Q) = 0$
    \item The maximum value of KL divergence is $+\infty$. When for any $x_i$, $Q(x_i)=0$, then $\log{\frac{Q(x)}{P(x)}} = +\infty$, $KL(P||Q) = +\infty$. 
    \item KL divergence is not symmetric. 
    \item   The equation is proved from right. According to definition:
        \begin{align*}
         KL(P(X,Y)||Q(X,Y))&=\sum_x\sum_yP(X,Y)log\frac{P(X,Y)}{Q(X,Y)}\\ 
         &=\sum_x\sum_yP(X,Y)log\frac{P(Y|X)P(X)}{Q(Y|X)Q(X)}\mbox{~(Chain rule)}\\ 
         &= \sum_x\sum_yP(X,Y)log\frac{P(Y|X)}{Q(Y|X)}+\sum_x\sum_yP(X,Y)log\frac{P(X)}{Q(X)}\\ 
         &= \sum_x\sum_yP(Y|X)P(X)log\frac{P(Y|X)}{Q(Y|X)}+KL(P(X)||Q(X))\\ 
         &= \sum_xP(X)\sum_yP(Y|X)log\frac{P(Y|X)}{Q(Y|X)}+KL(P(X)||Q(X))\\ 
         &= KL(P(Y|X)||Q(Y|X))+KL(P(X)||Q(X))\\
         QED
        \end{align*}
    \item In this problem, we'd like to find parameter $\theta$ to maximize the distribution $P_\theta(x) =P(x|\theta)$. Let $\hat{P}$ be the empirical distribution, we have:
        \begin{align*} 
        KL(\hat{P}||P_\theta)&=\sum_x\hat{P}\log\frac{\hat{P}}{P_\theta}\\
        &=\sum_x\hat{P}\log\hat{P}-\sum_x\hat{P}\log{P_\theta}\\
        &=-H(\hat{X})-\sum_x\hat{P}\log{P_\theta}
        \end{align*}
        Since $H(\hat{X})$ and $\hat{P}$ are irelevant with $\theta$ and are all greater or equal to 0, therefore,
        $$\arg{\min_\theta}KL(\hat{P}||P_\theta)=\arg{\max_\theta}\sum_x\log{P_\theta}$$
        QED
\section{K-means}
\section{K-medoids}
\begin{itemize}
    \item $\mathfrak{Advantages}$
        \begin{enumerate}
            \item K-medoids is more robust to noise and outliers.

        \end{enumerate}
    \item $\mathfrak{Disadvantages}$
        \begin{enumerate}
            \item K-medoids is harder than K-means, i.e. its time complexity is higher than K-meas.
        \end{enumerate}
\end{itemize}
    
                    


                
\end{enumerate}


\end{document}
